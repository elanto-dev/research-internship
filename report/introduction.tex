\chapter{Introduction}\label{introduction}

\glspl{llm} are statistical models that were trained on the large amount of data containing billions of parameters and use previous tokens to predict the future tokens~\cite{gpt-usage-and-lim}. \gls{llm} are used in different domains such as conversation agents, education, text summarisation, explainable AI, information retrieval and many more~\cite{Dale_2021}. One of the most well-known models is \gls{gpt}~\cite{Floridi2020-gpt3paper}, a chatbot developed by OpenAI and first released in November 2022. According to OpenAI CEO Sam Altman, \gls{gpt} reached 100 million weekly active users in November 2023\footnote{https://www.theverge.com/2023/11/6/23948619/openai-chatgpt-devday-developer-conference-news}. 

\gls{gpt} drew so much attention for its detailed and
expressive answers across many domains of knowledge, including software development. The chatbot was adopted by some software engineers as an alternative to using Google for solving software development related issues. Studies have investigated \gls{gpt}'s efficacy when used in software development processes~\cite{Hörnemalm_2023}, its limitations~\cite{gpt-usage-and-lim}, domains of application within issue-tracking systems and reliance on the \gls{gpt}-generated code~\cite{gpt-issue-tracking} and the correctness of the generated code~\cite{gpt-code-correctness, zhang2023critical}. Additionally, some studies focus on the human-bot interaction: the role of
\gls{gpt} in the process of collaborative software architecting~\cite{gpt-software-architecture} or what is \gls{gpt}'s primary use in issue-tracking systems~\cite{gpt-issue-tracking}. 

In this study we focus on the domain of human-bot interaction and use developer prompts to extract the different types of collaboration that developers want \gls{gpt} to follow during their interaction. We expect to see requests for \gls{gpt} to be a pair programmer and work together on issue resolution, where developer provides a feedback on how the chatbot answer can be improved and vice-versa; or it will be asked to review the code and suggest improvements or find a bug in the program. There are many ways developers can collaborate with \gls{gpt} and we focus on extracting some patterns from their prompts to recognise what type of collaboration does developer expect from \gls{gpt}. This leads to the research question of this study: \textbf{What role does \gls{gpt} take in user-chatbot interaction with software developers?}

To answer the research question we use various supervised and unsupervised text mining techniques: n-grams extraction, sequence mining and topic modelling. These methods help us with understanding the common themes and topics present in developers' \gls{gpt} prompts and the expectations they place on \gls{gpt} in these prompts. Sequence mining and topic modelling provide information extracted from the dataset that describes the data, while n-grams is used to provide the initial overview of what are the most common words/word combinations present in the data we are working with. We apply the mentioned techniques on DevGPT~\cite{devgpt} dataset that consists of developer-\gls{gpt} conversations sourced from GitHub\footnote{\url{https://github.com}} commits, pull requests, code files, issues and discussion, and from HackerNews\footnote{\url{https://news.ycombinator.com}}. 

The answer to this research question provides valuable insights into the role that \gls{gpt} plays in developer-chatbot collaboration. Additionally, it shows what tasks developers trust chatbot like \gls{gpt} to perform and what expect from \gls{gpt} in that role. By outlining the collaboration patterns of developer-chatbot interaction, we provide the ground for new research to develop or enhance chatbots' ability to help the software developer in the way that is the most expected from the chatbot.

We firstly provide some background information and related research in Chapter~\ref{related-work}. Then we describe in details what dataset consists of, what pre-processing steps are taken, how the data minins techniques are applied, provide some statistics, outliers and selected methods limitations in Chapter~\ref{methodology}. Chapter~\ref{results} focuses on describing and analysing the results. Chapter~\ref{conclusions} provides an overview of what was done and what results were achieved with this study, and provides the discussion of the project and possible future research. 
